\documentclass[12pt]{article}

\usepackage[margin = 30mm]{geometry}
\usepackage{abstract}
\usepackage{titlesec}
\usepackage{titletoc}
\usepackage{tikz}
\usetikzlibrary{trees}
\usepackage{fancyvrb}
\usepackage{xcolor}
\usepackage{amssymb}



\setlength\parindent{0pt}
\frenchspacing

\usepackage{hyperref}
\hypersetup{
    colorlinks,
    linktoc = all,
    citecolor=black,
    filecolor=black,
    linkcolor=black,
    urlcolor=black
}

\titleformat{\section}{\center\bf \large}{\thesection.}{0.5em}{}
\titleformat{\subsection}{\center\bf}{\thesubsection.}{0.5em}{}



\newcommand{\localtextbulletone}{\textcolor{black}{\raisebox{.35ex}{\footnotesize$\bullet$}}}
\renewcommand{\labelitemi}{\localtextbulletone}



\titlecontents{section}
	[0pt]
	{ }
	{\contentsmargin{0pt}\thecontentslabel.\enspace\normalsize}
	{\contentsmargin{0pt}\normalsize\MakeUppercase}
	{\titlerule*[.5pc]{}\contentspage}
	[]

\titlecontents{subsection}
	[15pt]
	{ }
	{\contentsmargin{0pt}\thecontentslabel\enspace\normalsize}
	{\contentsmargin{0pt}\normalsize\MakeUppercase}
	{\titlerule*[.5pc]{}\contentspage}
	[]





\setcounter{tocdepth}{1}

\title{Standardisation Guide for ``\texttt{rbf-tools}''}
\author{N.K. - \texttt{kraemer@ins.uni-bonn.de}}
\begin{document}
\maketitle
\begin{abstract}
The purpose of this guide is to collect any standardisation I have come up with over time. On top of that, I mention other guidelines, like naming conventions, structure conventions and more. 
\end{abstract}
\begin{figure}[h]
\centering
\begin{minipage}{0.5\textwidth}
\tableofcontents
\end{minipage}
\end{figure}




\section{Purpose of the module}

The purpose of the module is to collect most of the things I have programmed in the past 18 months with regard to radial basis functions. This collection is supposed to be handed over, eventually, without losing any reusability possibilities--i.e. I should not be the only one who understands this.\\

Resuability is a driving force of most of the modules. Many features have to be used in almost every script; for instance, building a kernel matrix. I got sick of doing it from scratch everytime, hence I started this collection.


\section{Hierarchy}
The hierarchy is supposed to be as flat as possible. The only things that are supposed to be in directories are figures and pointsets; see Figure \ref{fig:hierarchy}.

\begin{figure}[h]
\centering
\tikzstyle{every node}=[draw=black, thick,anchor=west]
\tikzstyle{module}=[draw=black, fill=blue!10!white]
\tikzstyle{script}=[draw=black, fill=green!10!white]
\tikzstyle{optional}=[dashed,fill=gray!50]
\begin{tikzpicture}[%
	grow via three points={one child at (0.5,-0.7) and
 	two children at (0.5,-0.7) and (0.5,-1.4)},
	edge from parent path={(\tikzparentnode.south) |- (\tikzchildnode.west)}]
	\node {rbf-tools}
    		child { node[module] {\texttt{kernelFcts.py}}}		
    		child { node[module] {\texttt{kernelMtrcs.py}}}
    		child { node[script] {\texttt{interpolMatern1d.py}}}
    		child { node {figures}}
   		child { node {pointsets}};
\end{tikzpicture}
\caption{File structure of \texttt{rbf-tools}; modules in blue, scripts in green}
\label{fig:hierarchy}
\end{figure}


\section{Naming and coding conventions}
I follow naming conventions with two purposes in mind:
\begin{enumerate}
\item Good programming practice
\item Unification
\end{enumerate}

\subsection{Good naming practice}
The following is a list of most naming conventions regarding good practices:
\begin{enumerate}
\item \textbf{Variable naming:} 
\begin{itemize}
\item \textbf{Descriptive naming:} do not use \texttt{x}, \texttt{N} or \texttt{K}, but \texttt{pt}, \texttt{numPts} or \texttt{kernelMtrx}
\item \textbf{Short names:} do not use \texttt{standardKernelMatrixWithMaternKernel}, but \texttt{kernelMtrx}
\item No underscores (privilege of python)
\item No all-uppercase variables (privilege of python)
\item Indicate new ``term'' with a single uppercase letter: \texttt{kernelFct}, \texttt{kernelMtrx}, \texttt{ptSet}
\end{itemize}
\item \textbf{Commenting:} As long as the variables are named well, I do not need comments except for very few occasions
\item \textbf{Function naming:} verb-noun scheme, i.e. \texttt{buildKernelMtrx}, \texttt{getPtSet}, ...
\item \textbf{File naming:} Each file has to include the following information:
\begin{enumerate}
\item \textbf{Name:} e.g. \texttt{'interpolation.py'}
\item \textbf{Purpose:} Describe the purpose of the file in a single sentence (if that is not possible, think again about starting this file at all)
\item \textbf{Description:} Describe the method in two or three sentences giving the main keywords
\item \textbf{Author:} Usually me
\end{enumerate}
An exemplary header is the following, taken from \texttt{'interpolMatern1d.py'}:
\begin{Verbatim}[formatcom=\color{blue!50!black}]
# NAME: 'interpolMatern1d.py'
#
# PURPOSE: Basic 1-dimensional interpolation using Matern functions
#
# DESCRIPTION: I solve a system involving a Matern-kernel matrix 
# where the Matern kernel is based on scipy.special's functions
# and plot the 10-th Lagrange function.
#
# AUTHOR: NK, kraemer(at)ins.uni-bonn.de
\end{Verbatim}
\end{enumerate}

\subsection{Unification}
The following is a list of most naming conventions regarding a unified system:
\begin{enumerate}
\item \textbf{Kernel functions:} I refer to kernel functions and kernel matrices using \texttt{kernel}, not \texttt{kern} nor \texttt{cov}
\item \textbf{Common Abbreviations:} I use as common abbreviations:
\begin{itemize}
\item Indices: \texttt{idx}, \texttt{jdx}, \texttt{kdx}, ...
\item Point: \texttt{pt}
\item Pointset: \texttt{ptSet}
\item Numer of points: \texttt{numPts}
\item Matrix: \texttt{mtrx}, matrices: \texttt{mtrcs}
\item Length of a vector called \texttt{vecAbc}: \texttt{lenVecAbc}
\item Pointset for evaluation (plotting): \texttt{evalPtSet}
\item Number of evaluation points: \texttt{numEvalPts}
\item Lebesgue constant: \texttt{lebCnst}
\item Gaussian: \texttt{gauss} (as in \texttt{gaussKernel} instead of \texttt{gaussianKernel})
\end{itemize}
\end{enumerate}




\subsection{Other good practices}
\begin{enumerate}
\item \textbf{Functions:}
\begin{itemize}
\item Each function should serve \textbf{a single} purpose which should be clear from the naming
\item Each function should be deterministic, i.e. two runs with the same input give the same output (this type of function is called pure function). In my case this often depends on random numbers; see next point
\end{itemize}
\item \textbf{Seeds for random numbers:} Each file should always give the same result as long as nothing is changed. Hence, start everything that includes random numbers with \texttt{np.random.seed(15051994)} and do not set another seed elsewhere
\item Readability of a program often trumps performance
\end{enumerate}

\section{Modules}

In the following I describe some module files and their conventions.



\subsection{Kernel functions}
	\label{subsec:kernelFcts}
I collect kernel functions in the file \texttt{kernelFcts.py}. They all take two points as inputs and give out a scalar. As an example, the Gaussian:
\begin{Verbatim}[formatcom=\color{blue!50!black}]
def gaussKernel(ptOne, ptTwo, lengthScale = 1.0):
    distPts = np.linalg.norm(ptOne - ptTwo)
    return np.exp(-distPts**2/(2*lengthScale**2))
\end{Verbatim}

The distance of the two inputs, \texttt{ptOne} and \texttt{ptOne}, is computed inside the function. The purpose of this is that I can construct kernel matrices in a very easy manner; see subsection \ref{subsec:kernelMtrcs}

\subsection{Kernel matrices}
\label{subsec:kernelMtrcs}
I collect kernel matrices in the file \texttt{kernelMtrcs.py}. The all take two pointsets as inputs and return a matrix. As an example, the standard kernel matrix: 
\begin{Verbatim}[formatcom=\color{blue!50!black}]
def getKernelMtrx(ptSetOne, ptSetTwo, kernelFct):
    lenPtSetOne = len(ptSetOne)
    lenPtSetTwo = len(ptSetTwo)
    kernelMtrx = np.zeros((lenPtSetOne, lenPtSetTwo))
    for idx in range(lenPtSetOne):
        for jdx in range(lenPtSetOne):
            kernelMtrx[idx,jdx] = kernelFct(ptSetOne[idx,:], ptSetTwo[jdx,:])
    return kernelMtrx
\end{Verbatim}

The input pointsets need to have the same dimension, but do not need to match in size. The kernel function \texttt{kernelFct} needs to be of the form I described in subsection \ref{subsec:kernelFcts}

\end{document}